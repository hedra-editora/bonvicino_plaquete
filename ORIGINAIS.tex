\part{LADO A}

\chapter*{}
\addcontentsline{toc}{chapter}{A nova utopia (1), Lida por Régis Bonvicino}
\textbf{A nova utopia (1)}\\
Lida por Régis Bonvicino

\bigskip

A nova utopia é uma borboleta negra, desatenta, com olhos exuberantes. A
nova utopia é a favor da proteção implacável dos animais. A nova utopia
é inclusiva, participativa. A nova utopia é o coro afinado dos
descontentes. É um ex-guerrilheiro, de porte avantajado, homem forte do
governo. A nova utopia tem informações privilegiadas, disponíveis. É um
ex-leproso. A nova utopia rechaça a figura de Nossa Senhora se
masturbando. A nova utopia defende os direitos das trabalhadoras do
sexo. A nova utopia comunga, com moderação, ideais materialistas. A nova
utopia morre de pé. É, ao mesmo tempo, um \emph{duty free} e um
\emph{detox} financeiro. A nova utopia é nosso dever como cidadãos. A
nova utopia exalta a sustentabilidade das empresas. A nova utopia sabe
que se pode ser árabe e muçulmano, árabe e não muçulmano, muçulmano sem
ser árabe. Negro sem ser branco, branco sem ser negro. A nova utopia é a
liberdade de expressão do \emph{Le Monde}, reassegurada desde sempre. A
nova utopia é um ajuste de contas contra o obscurantismo dos outros. A
nova utopia rejeita factoides politicamente úteis. A nova utopia é um
pouco xiita, apenas quando estritamente inevitável. É um turista
americano visitando o Museu Abu Ghraib. A nova utopia tem logo e
\emph{slogan}. Condena chacinas na periferia. A nova utopia emite notas
de repúdio, lança abaixo-assinados; defende o grafite; a nova utopia
prega a bicicleta. A nova utopia é o respeito incondicional ao nanismo.
Condena corruptos. É um ex-ladrão. Tem seu próprio dicionário. Pensa
antes de agir. Repele palavras e pede ação. A nova utopia é um ex-coxo.
É a asa aberta do voo. É um \emph{showroom} de exuberâncias naturais. É
um céu com nuvens negras, sob controle. É uma estante de livros num
banheiro. É a viúva de Jorge Luis Borges detalhando seu processo de
criação. A nova utopia é um ex-macumbeiro, um ex-bêbado, é um ex-exu
sujo. É um branco de alma preta. A nova utopia é ainda o indígena de
tocheiro, fazendo política, diariamente, nas redes sociais. A nova
utopia é uma ex-esteticista de unhas postiças. É um espião trans pegando
sol num roteador. É um ex-selvagem. É uma ex-vadia. É um ex-puto. É uma
entendida. É um ex-pária. É uma míriade de franquias de poetas
premiados. É um poema à altura de seu tempo.

\pagebreak

\textbf{Vinheta 1:}
\addcontentsline{toc}{chapter}{Vinheta 1, Lida por Régis Bonvicino}

``Boa noite. Deus devolve o revólver''

Lida por Régis Bonvicino

\pagebreak

\textbf{Ficção}
\addcontentsline{toc}{chapter}{Ficção, Lida por Régis Bonvicino}

Lido por Régis Bonvicino

\begin{verse}
Pendurada de cabeça para baixo\\
via ao contrário a coluna em estilo coríntio\\
da antiga estação de trem\\
o poder político dos estoques\\[5pt]
as sacas de café, Bolsa de Nova York,\\
camionetes queimadas\\
um som adocicava o Largo\\
talvez fosse o de uma nova canção dos Beatles\\[5pt]
``Metralhado e morto outro facínora''\\
você com essa cara de filha de Maria\\
uma paulada na coluna\\
quebra as vértebras dessa puta\\[5pt]
boca fechada, o aparelho intacto\\
\emph{flashback} íntimo\\
o apontamento entre as páginas de um livro\\
o porta-malas do camburão\\[5pt]
as manchetes nas bancas\\
``O PIB vai a 10\%'',\\
``Prisioneiros viajam hoje'', alívio\\
baratas na vagina\\[5pt]
corte, tesoura, um talho no sutiã\\
tesoura roçando os seios\\
pode pisar neles\\
barata devidamente arquivada no cu\\[5pt]
o capuz, cabeça enfiada na água suja\\
os gritos sem porrada\\
ramos de café perfilam o capitel\\
portas maciças da altura do edifício Itália\\[5pt]
as teclas da pianola\\
em outro andar, mãos algemadas\\
nu na cadeira do dragão\\
o corpo do cara ficou odara
\end{verse}

\pagebreak

\textbf{Sermão}
\addcontentsline{toc}{chapter}{Sermão, Lida por Régis Bonvicino}

Lido por Régis Bonvicino

\begin{verse}
Uma \emph{joint-venture} de sem tetos, pele e osso, com medo da polícia, \qb{}aglomerada\\
no centro histórico de São Luís, mais de meia-noite, um dos mendigos, \qb{}bêbado, diz\\[15pt]
palavras ao vento\\
botaram Ford Landau no convento\\
padre não prega mais de costas --\\
contra -- como Vieira, mas de joelhos\\[5pt]
me ouçam logo de novo os peixes\\
tambor também não põe a mesa\\
galinha morta é veneno\\
o boi-bumbá só tem cordeiro\\[5pt]
para que serve tanto azulejo?\\
botaram mais um ladro no governo\\
urubu bica pedregulho\\
mato nasce em telhado\\[5pt]
o duque pega tudo\\
é bem mais de 10\%\\
lenda é passatempo\\
ganja é coisa de regueiro\\[5pt]
o mendigo há de subir\\
com o \emph{crack} na cabeça\\
moeda aqui cai somente no bueiro\\
escadaria é monumento\\[5pt]
miséria aqui é matinê\\
é bom guardar segredo\\
cavalo\\
não defeca mais dinheiro
\end{verse}

\pagebreak

\textbf{Vinheta 2:}
\addcontentsline{toc}{chapter}{Vinheta 2}


\begin{verse}
``A mão que afaga é a mesma que apedreja\\
Apedreja essa mão vil que te afaga''\\[5pt]
Lida por Régis Bonvicino
\end{verse}

\pagebreak

\textbf{Interlude}, Rodrigo Dário
\addcontentsline{toc}{chapter}{Interlude, Rodrigo Dário}

\pagebreak

\textbf{Álibi}
\addcontentsline{toc}{chapter}{Álibi, Lida por Régis Bonvicino}

Lido por Régis Bonvicino

\begin{verse}
Oh, Pai, tende piedade\\
dos zilionários, dos vendedores legais de armas\\
dos lobistas, do dinheiro farto dos narcos\\
dos unhas de fome, dos gigolôs dos cassinos\\
dos traficantes de iguanas, rim e fígado\\[5pt]
Oh, Pai, tende piedade\\
dos banqueiros, dos juros sobre juros,\\
do laissez-faire chinês, do marketing do bem\\
dos plutocratas, dos fundos-abutres\\
garras, o condor-dos-andes não canta\\[5pt]
Oh, Pai, tende piedade\\
dos meões do dinheiro sujo dos contratos públicos\\
daqueles que depreciam os papéis de P.P. Pasolini\\
daqueles que lavam dinheiro com H. Matisse\\
misericórdia divina, delícia e êxtase dos santos\\[5pt]
Oh, Pai, tende piedade\\
dos xeques, dos grandes proprietários de terra\\
daqueles que não entregam a lebre\\
dos traficantes de marfim, caveiras com dentes e pedras\\
da criptomoeda, dos chefetes políticos despóticos\\[5pt]
Oh, Pai, tende piedade\\
dos traficantes de lixo eletrônico, dos agiotas\\
dos matadores de aluguel, dos guarda-costas\\
dos sócios ocultos, dos donos de \emph{offshores}\\
Oh, Pai, sobretudo tende piedade de nosso honrado \emph{boss}.
\end{verse}

\pagebreak

\textbf{Tarde}
\addcontentsline{toc}{chapter}{Tarde, Lida por Régis Bonvicino}

Lido por Régis Bonvicino

\begin{verse}
Junto ao muro arbustos, cactos\\
um deles, pontiagudo, alto,\\
avança na calçada\\
um grupo de nuvens cinza pesa\\
no ombro de uma mendiga negra que passa\\
poste, fios, janela, uma gambiarra\\
um carro pifado na guia\\
a um passo da avenida\\
parede de cubos de vidro\\
edifício, canteiro de espinhos,\\
dorme\\
estirado a uma certa distância da entrada\\
cabeça sobre a garrafa vazia de água\\
o sol de inverno bate direto em sua cara\\
roupas do corpo,\\
sem sapatos,\\
não tem mais nada, não tem \emph{spleen}\\
só tem porrada
\end{verse}

\pagebreak

\textbf{Perspectiva}
\addcontentsline{toc}{chapter}{Perspectiva, Lida por Régis Bonvicino}

Lido por Régis Bonvicino

\begin{verse}
Muro baixo do cemitério\\
um galho da tipuana atravessa o arame farpado,\\
túmulos à vista, altos\\
duas cruzes de mármore\\[5pt]
do lado oposto da rua\\
um cara estirado na calçada\\
debaixo das grades da janela do térreo\\
o motorista dá a partida\\[5pt]
um casal: o marido empurra o carrinho do bebê\\
pessoas entram no edifício de tijolos à vista\\
na pequena casa geminada: consertos rápidos,\\
costureira na máquina\\[5pt]
um gavião pousa numa antena\\
o cara acorda\\
olha para as grades, mija na parede,\\
mais uma loja fecha\\[5pt]
a de aluguel de fantasias,\\
roupas para teatro e cinema\\
um mendigo se apaga nessas linhas\\
outro reaparece na cena
\end{verse}

\pagebreak

\textbf{Retrato}
\addcontentsline{toc}{chapter}{Retrato, Lida por Caroline De Comi e Régis Bonvicino}

Lido por Caroline De Comi e Régis Bonvicino

\begin{verse}
Uma carroça cruza a pista\\
carregada de garrafas, caixas, cabos, um motor\\
Sábado à tarde, fim de verão,\\
lojas fechadas\\[5pt]
o sol bate nos letreiros\\
cachorros disputam um saco de lixo\\
ônibus passam, meio vazios\\
um motorista para no ponto de táxi\\[5pt]
vestidos de núpcias, vitrines,\\
Miss Luxúria\\
gambiarras nos postes\\
fios atravessam a copa de uma goiabeira\\[5pt]
na esquina da rua Oriente\\
com a rua Casemiro de Abreu\\
nuvens,\\
o mormaço, atrás das folhas, se atenua\\[5pt]
uma única flor,\\
pétalas brancas, estames amarelos,\\
abrupta\\
uma letra pende\\[5pt]
do alto da porta de uma loja\\
um mendigo dorme\\
cabeça largada na mureta do canteiro\\
goiabas apodrecem em autópsia mútua
\end{verse}

\pagebreak
\textbf{Trailer}
\addcontentsline{toc}{chapter}{Trailer, Lida por Régis Bonvicino}

Lido por Régis Bonvicino

\begin{verse}
Um cara descarta o resto do sanduíche\\
recostada no pé da lixeira\\
um pedaço de pão cai na cabeça da mendiga,\\
um outro cara joga um maço de cigarros vazio\\
a mulher é negra,\\
umas garotas, lata de pepsi, casca de sorvete\\
lojas, o logo do banco, câmeras\\
um cara atravessa na faixa\\
ela pede esmola:\\
``Eu não sou artista''\\
o executivo olha para o outro lado da avenida\\
um camelô entra na calçada\\
o dia porra tem que valer a pena\\
no quiosque, a manchete:\\
``O desemprego aumenta''\\
um obeso mórbido passa,\\
camiseta branca, encharcada de suor,\\
a raiz do fícus força as bordas do canteiro\\
sem nenhum puto entre os dedos\\
a câmera pifa, ela sai de cena
\end{verse}

\pagebreak

\textbf{Lápide}
\addcontentsline{toc}{chapter}{Lápide, Lida por Régis Bonvicino}

Lido por Régis Bonvicino

\begin{verse}
Os soluços longos dos violinos do outono\\
aqui Rimbaud\\
aquele otário\\
te enrabou por uns trocados
\end{verse}


\part{LADO B}

\chapter*{}
\textbf{Haiku}\\
\addcontentsline{toc}{chapter}{Haiku, Lida por Régis Bonvicino}
Lido por Régis Bonvicino

\begin{verse}
Pedra no cachimbo\\
Estação da Luz: porrada\\
Verão, sol lilás\\[5pt]
Pedra, narguilé\\
Doce como mel: porrada\\
Verão, o sol âmbar\\[5pt]
É o Incrível Hulk\\
Um avião nos pés: porrada\\
Janeiro, sol púrpura\\[5pt]
Uns tragos na lata\\
De asas já nos pés: porrada\\
Março, sol turquesa\\[5pt]
Cachimbo, cristal\\
Braços alados, porrada\\
Março, um raio fúcsia\\[5pt]
Lata sem anel\\
O anu bica o olho do noia\\
Isqueiro na dobra\\[5pt]
Pedra no cachimbo\\
Arco-íris nos pés, porrada\\
Dezembro, sol sépia\\[5pt]
Canudo, Yakult\\
Mãos lixam o céu, porrada\\
Março, sol magenta\\[5pt]
Cachimbo na roda\\
Garras de tigre, porrada\\
Janeiro, sol jade\\[5pt]

\pagebreak

Em nome de Buda,\\
Nada obstante uma brisa\\
Verão, sol sem cor\\[5pt]
Cavalo, porrada\\
O tubo de pvc\\
Outono, sol ágata
\end{verse}

\pagebreak

\textbf{Vinheta 3:}
\addcontentsline{toc}{chapter}{Vinheta 3}

Deus devolve o revólver

\pagebreak

\textbf{Luz}
\addcontentsline{toc}{chapter}{Luz, Lida por Régis Bonvicino}

Lido por Régis Bonvicino

\begin{verse}
Sucateiro rastafari\\
sentado na mureta, cabeça baixa\\
sob as palmeiras do largo\\
pés na mochila, garrafas pet\\[5pt]
\emph{player} do ecossistema global\\
um rato entra no bueiro.\\
O relógio da Luz\\
sob um sol de rachar\\[5pt]
daqui é apenas uma torre\\
o vapor sobe do asfalto.\\
Cicatriz na cara da puta\\
pista dupla, atravessa a avenida\\[5pt]
\emph{short} verde, blusa regata\\
cabelo curto, o michê esfria\\
os muros exalam um cheiro de urina.\\
A noite abate o dia\\[5pt]
um cachorro fareja, tranquilo,\\
a calçada limpa,\\
outro, órfão de um noia,\\
uiva na esquina.
\end{verse}

\pagebreak

\textbf{Interlude}
\addcontentsline{toc}{chapter}{Interlúdio, por Rodrigo Dário}

Por Rodrigo Dário

\pagebreak

\textbf{\emph{Hic Jacet Lepus}}
\addcontentsline{toc}{chapter}{Hic Jacet Lepus, Lida por Régis Bonvicino}

Lido por Régis Bonvicino

\begin{verse}
Perto de uma sinagoga, enquanto\\
o faxineiro varre a entrada do prédio,\\\
um catador, velho, de barba rala,\\
pega uma latrina na caçamba\\[5pt]
Outro catador, boné branco,\\
``ulalá'' grafado em azul acima da aba,\\
dois dentes podres à vista,\\
repete em voz alta: ``o lixo é sujo''\\[5pt]
Na banca, uma tevê: ataques na Síria\\
bombas na cara dos civis\\
Outro sucateiro, de mãe talvez zíngara,\\
saco plástico preto, aberto, percorre a calçada\\[5pt]
latas vazias de Pepsi, Coca, cerveja\\
pede na lanchonete, no \emph{self-service}, no bistrô\\
Na saída do \emph{shopping} militantes coletam assinaturas para\\
um manifesto em favor das abelhas,\\[5pt]
ágora na hora da xepa,\\
um santinho da Virgem colado no poste\\
um garoto negro, quase no ponto de ônibus,\\
um par de baseados no bolso,\\[5pt]
é preso em flagrante, por tráfico?\\
leva porrada na rua, \emph{ora pro nobis}\\
camburão, algemas\\
deus devolve o revólver
\end{verse}

\pagebreak

\textbf{A nova utopia (2)}
\addcontentsline{toc}{chapter}{A nova utopia (2), Lida por Régis Bonvicino}

Lido por Régis Bonvicino

\begin{verse}
É um discurso estritamente atrelado à realidade\\
É um inferno fiscal\\
É uma empresa real\\
É o lado útil da palavra\\[5pt]
É uma brancaiada tola\\
É a nota mínima\\
É o aplicativo \emph{Equitable}\\
É um café da manhã sem lactose\\[5pt]
É a cerveja sem álcool e o cigarro eletrônico\\
É uma \emph{prece}, e não a Ave Maria, sussurrada\\
num beco de uma favela\\
É um sinônimo de vândalo\\[5pt]
É o alarme contra\\
o impacto ambiental de um passeio de barco\\
É um protesto contra aulas de inglês\\
É uma tr@b@lh@dor@ em transição de empregos\\[5pt]
É um supervilão asfixiando apenas machos\\
É um chá com escritores do \emph{website} e do jornal\\
É um \emph{pet} sem sobrepeso\\
É a queda programada da taxa de juros\\[5pt]
É um ladrão de galinhas\\
É um desvio de verbas públicas\\
É a vítima de um assalto\\
se desculpando com o assaltante\\[5pt]
É a redenção ecológica do joio\\
É um \emph{approach} jurídico para o diabo\\
É um cego paraolímpico\\
É um antiverso altamente subversivo\\[5pt]
É um alvo potencial do terrorismo linguístico\\
É um drácula \emph{hard-core} doando sangue\\
É o direito à segunda via garantido\\
É um morador de rua revirando uma lata de lixo seletivo\\[5pt]

\pagebreak

É o produto da venda legal de armas\\
Violinos afinam a brisa fétida\\
É uma lavagem de palavras\\
É uma filha convicta da pátria
\end{verse}

\pagebreak

\textbf{Áudio}
\addcontentsline{toc}{chapter}{Áudio, Lida por Caroline De Comi}

Lido por Caroline De Comi

\begin{verse}
O sol da manhã bate em sua cara\\
deitada no chão\\
rente à mureta do parque\\
fios de cabelo branco escapam da tiara\\[5pt]
cabeça sobre a bolsa\\
em frente à torre do relógio da estação\\
hibiscos vermelhos:\\
renques ao longo das grades\\[5pt]
mão sobre o rosto\\
talvez ela tenha chegado no último trem da noite\\
talvez ela não esteja dormindo\\
talvez ela esteja sem clientes\\[5pt]
vestido longo cinza\\
sapatos baixos, pele seca dos pés\\
talvez ela esteja a caminho do emprego\\
talvez ela vá pegar o metrô\\[5pt]
a polícia aqui não mata todos os dias\\
ao fundo palmeiras em linha\\
mendigo negro, cabeça baixa,\\
de novo, sentado na guia\\[5pt]
um ambulante vende água\\
talvez ela tenha escrito os versos:\\
``desatenta, fui castigada,\\
passei a vida ao largo''.\\[5pt]
talvez ela tenha feito algum dinheiro\\
talvez ela seja figurante de um filme\\
talvez ela seja um cartaz perdido\\
a luz, rasante, incide sobre as rugas de seu rosto\\[5pt]
a mandíbula de uma arara\\
um gavião pousa no topo de um cedro\\
mais alto que os prédios\\
talvez ela seja um acará ou uma carpa\\[5pt]

\pagebreak

espelhos d'água\\
uma andorinha, fosforescente, sobrevoa a grade\\
talvez ela não seja mais que um efeito de arte\\
talvez ela não passe de um \emph{close-up}
\end{verse}

\pagebreak

\textbf{Da janela do quarto}
\addcontentsline{toc}{chapter}{Da janela do quarto, Lida por Régis Bonvicino}

Lido por Régis Bonvicino

\begin{verse}
Manhã, janela do quarto do hotel:\\
o carroceiro puxa a carroça\\
uma van da Transcootour passa\\
pés descalços no asfalto\\[5pt]
gaivotas e urubus\\
se encaram por lixo, peixes e céu\\
banhistas em sua rotina mecânica de\\
sol, areia e ginástica\\[5pt]
duas putas insones roçam\\
os peitos nos vidros\\
de um Honda Civic.\\
A garota de Ipanema\\[5pt]
de Vinícius e Tom Jobim\\
mora hoje em Arrelia, Andaraí\\
é mais que um poema\\
paga o dízimo, da igreja e da milícia,\\[5pt]
ônibus lotado, caindo aos pedaços,\\
de moto, um PM arranca o celular\\
hoje pelo menos faltou presunto\\
a caminho do mar,\\[5pt]
e sua irmã gêmea, a coisa mais linda,\\
mora com o gigolô da boca\\
em Drummond, Cachambi\\
Papelotes de cristal e cocaína no sutiã\\[5pt]
à noite, frequenta o Leblon\\
Troca de tiros na \emph{web} e na tevê\\
Complexo da Maré,\\
Bossa Nova \emph{nightmare}
\end{verse}

\begin{quote}
Publicado em \emph{Le Monde Diplomatique}\\
sob o título ``Copacabana classic'', 1º de dezembro de 2008,\\
reescrito em 2018
\end{quote}

\pagebreak

\textbf{\emph{The new utopia} (1)}
\addcontentsline{toc}{chapter}{\emph{The new utopia (1)}, Lida por Charles Bersntein}

Lido por Charles Bersntein

\pagebreak


\textbf{A nova utopia (7)}
\addcontentsline{toc}{chapter}{A nova utopia (7), Lida por Caroline De Comi}

Lido por Caroline De Comi

\bigskip

\emph{To dream has been the business of my life.}~A nova utopia~mata por
adição e, entre outras coisas, por agonismo opioide. Faz seu inimigo
sentir náuseas, sonolência, vertigem, vômitos, dor de cabeça, pressão
baixa e choque. Faz seu inimigo falar com voz de lixa: ela não brinca em
serviço. A nova utopia não é um despropósito.~O novo utopista não é um
dândi da sombra. É o drone suicida roubado da Kalashnikov: carrega
explosivos e~abre caminhos na selva, na favela, nas cidades. Tudo pela
causa. Também ecológica, é a favor de cocaína batizada com pó de
nenúfar. É a favor, unicamente, de obras como as de Shakespeare a cada
manhã. A nova utopia -- mirando um futuro melhor -- faz tráfico limpo. O
raio de sol só aparece à noite, para desanuviar. Pense, pense, você está
na Terra: não há remédio. Fique tranquila, numa boa. Tantos miligramas
de Depakote, tantos de Topiramato, a bula do Aristab -- a rotina diária
sem tréguas de pílulas por uma década. Comece por Zyprexa. Um robô morre
em Marte. Um cara atira, entre prateleiras, contra a própria cabeça
dentro da farmácia. Nada, absolutamente nada, mais um saco de merda sem
qualquer mérito.~Má notícia não existe.~Dinheiro não fede.~A chuva, a
chuva, a tempestade, uma enxurrada de lixo sobe dos bueiros às ruas. Sob
a marquise, um bispo se esgoela num megafone: ``botar um baseado nos
lábios é como fazer sexo oral para o diabo''. Só pode ser o trecho de um
filme, ouça o refrão da música: ``eles disseram que o inferno ferve''.~O
novo utopista é um sicário \emph{indie}. É chefe e servo de si mesmo. É
uma fera. Atira de bate-pronto nos sequestradores de cérebros e nos
revendedores de memória, que operam o mercado negro. Mata de verdade.~É
a favor do \emph{copyright}. É contra o \emph{copyleft}. O novo utopista
aplica doses pesadas de morfina nos algoritmos. Para ele, o algoritmo
atropela o trabalho. Outra cena? \emph{Crack} é a vitória. Um mendigo
puxa uma pedra na lata amassada de Coca-Cola. O novo utopista batalha
pela liberdade entre muros. Pede a benção à deusa~guarani Jururá-Açú
para assassinar o diabo do velho utopista.~À semelhança dela, entra e
sai do inferno como quem troca de camisa.~Não faz papel de morto
coadjuvante.~O ódio move mais do que qualquer programa político.~Só pode
ser mesmo um filme. Olhe o décor.~Trabalha também com~fungos
alucinógenos e LSD, para fazer mais grana para a luta.~É um ponto de
partida. Por que dizer isto aqui? Cai o letreiro: tantos miligramas de
Quetiapina, o antipsicótico atípico,~as bulas, o Latuda: para que
desespero? Johnson \& Johnson, Pfizer, Merck \& Co, Bayer, Novartis, EMS
Corp, Roche holding, Libra. Um grito talvez em \emph{off} se ouça no
cenário. Uma garota se joga do topo do edifício. Sonhar, sonhar, sonhar,
é o grande negócio da vida!

\pagebreak

\begin{vplace}[1]
\noindent{}Poemas do livro inédito \emph{A nova utopia}, de Régis Bonvicino\\
Leitura dos poemas: Caroline De Comi, Charles Bernstein e Régis Bonvicino\\
Produção, sonorizações, mixagem e capa: Rodrigo Dário\\
Edição e captação de som: Tamires~Pistoresi\\
Agradeço a Roque Andrade ter acolhido meu projeto de gravar os poemas.\\
Agradeço-lhe também ter me apresentado a Rodrigo Dário. RB\\
Gravado em São Paulo, de março a julho de 2019, no estúdio espaço-som, Rua Teodoro Sampaio, 462 e 512, Pinheiros -- CEP: 05405-050
\end{vplace}